\chapter{深度学习平台——TensorFlow}
\chaptermark{深度学习平台——TensorFlow}
	%GBrowse
	\section{TensorFlow简介}
	TensorFlow在2015年11月由Google开放,从此,它已经成为GitHub上最受欢迎的机器学习库。TensorFlow——第二代分布式机器学习算法实现框架和部署系统,是基于使用DisBelief时的经验及训练大规模分布是神经网络的需求开发的。但是在某些基准上,TensorFlow是DistBelief的两倍。可以方便地部署到各种平台,大大简化了真实场景中应用机器学习的难度。TensorFlow的计算可以表示为有状态的数据流式图,使用数据流式图来规划计算流程,可以将计算映射到不同的硬件和操作系统平台。对于大规模的神经网络训练,TensorFlow可以让用户简单地实现并行计算,同时使用不同的硬件资源进行训练,同步或异步地更新全局共享的模型参数和状态。
	TensorFlow是相对高阶的机器学习库,用户可以方便的用它设计神经网络结构,支持自动求导,核心代码是用C++编写的,简化了线上部署的复杂度,可以在手机和CPU这种内存紧张的设备上运行。除了C++接口外还有Python、Go、Java等接口。可以部署在一台或多台CPU、GPU上,兼容多个平台,包括Windows、Linux、Android等。有TF.Learn和TF.Slim等上层组件可以帮助快速的设计新网络,并且兼容Scikit-learn estimator接口,同时TensorFlow不局限于神经网络,数据流式图支持非常自由的算法表达,只要可以将计算表达成计算图的形式,就可以使用TensorFlow。
	另一个重要特点是它灵活的移植性,可以将同一份代码几乎不经过修改就轻松的部署到任意数量CPU或GPU的PC、服务器或者移动设备上。还有一个优势就是极快的编译速度,还有功能强大的TensorBoard,能可视化网络结构和训练过程,对于观察复杂的网络结构和监控长时间、大规模的训练很有帮助。
		%UCSC Genome Browser
		\section{核心概念}
		\subsection{计算图}
		UCSC Genome Browser是由University of California Santa Cruz (UCSC) 创立和维护的,该站点包含有人类、小鼠和大鼠等多个物种的基因组草图,并提供一系列的网页分析工具。站点用户可以通过它可靠和迅速地浏览基因组的任何一部分,并且同时可以得到与该部分有关的基因组注释信息,如已知基因,预测基因,表达序列标签,信使RNA,CpG岛,克隆组装间隙和重叠,染色体带型,小鼠同源性等。用户也可以因为教育或科研目的加上他们自己的注释信息。UCSC Genome Browser目前应用相当广泛,比如Ensembl 就是使用它的人类基因组序列草图为基础的。
		\subsection{可视化方式}
		JBrowse用 track 的方式进行可视化,提供平滑的动态移动和缩放功能,也有导航和通道的选择。JBrowse可以展示多种 track 视图,除基本视图外,还可以显示非翻译区、外显子、内含子结构等。
		
		\subsection{可视化内容}
		UCSC基因组浏览器提供了多样化的注释数据集(称为“轨迹”并以图形方式呈现),包括mRNA比对,DNA重复元件的映射,基因预测,基因表达数据,疾病关联数据(代表基因的关系疾病)和市售基因芯片(例如Illumina和Agilent)的映射。显示的基本范例是在水平维度上显示基因组序列,并显示mRNA的位置,基因预测等的图形表示。沿着坐标轴的颜色块显示各种数据类型对齐的位置。在单个坐标轴上显示这种大量数据类型的能力使浏览器成为数据垂直整合的便利工具。
		UCSC浏览器与其他基因组浏览器区分开来的一个独特而有用的功能是显示器的不断变化的性质。可以显示任何大小的序列,从单个DNA碱基到整个染色体(人类chr1 = 2.45亿碱基,Mb)和完整的注释轨迹。研究人员可以显示单个基因,单个外显子或整个染色体带,显示数十个或数百个基因以及许多注释的任意组合。方便的拖放功能允许用户选择基因组图像中的任何区域,并将其扩展到占据全屏。
		
		\subsection{系统架构}
		UCSC Genome Browser 的开发,起源于一小段应用于 C. elegans 基因预测拼接图谱的 C 语言脚本,后期通过不断扩充,才变成现在这样强大的一个分析工具。 现在 UCSC 的主要开发语言是Java/Python,后台数据库依赖于 mysql,而且提供mysql 的公共接口,只要用户本地电脑装有 mysql客户端,就可以通过 UCSC 提供的接口访问网站后台的数据库;对于前台要求,UCSC 可以较好地兼容 IE、Chrome、Firefox 等主流网络浏览器。UCSC 是完全开源的,用户可以下载到完整源码。
		\subsection{运行机理}
		\subsection{优缺点}
		优点:JBrowse 属于新一代基因组浏览器,作为GBrowse的继任者,是基于最新的前端技术开发的。在 JBrowse 中,服务器端的负荷极大地降低,后台服务器只需要向浏览器客户端发送数据文件,将繁杂的计算工作从服务端脱离出来,大量计算工作被合理分配到了前端。同时,JBrowse对Cookies技术也得到了很好的支持,可以有效记录用户的喜好。\\
		\indent 缺点: JBrowse 把可视化主要工作放在了浏览器端,但其可视化方法仅是一些普通的HTML 标签实现,造成可视化不友好等问题。同时浏览器在绘制图像时需要运行大量的JavaScript代码,而且目前主流浏览器对 HTML5 中新标签的支持不完善,造成用户体验不佳等问题。